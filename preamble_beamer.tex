\usepackage{tikz}
\usepackage{amssymb}
\usepackage{graphicx}
\usepackage{wasysym}
\usepackage{spverbatim}
\usepackage{natbib}
\usepackage{dsfont}
\usepackage{lmodern}
\setcitestyle{square}
\usepackage{float}
\usepackage{amsmath}
\usepackage{amscd}
\usepackage{hyperref}
\usepackage{enumerate}
\usepackage{amsfonts}
\usepackage{amssymb}
\usepackage[utf8]{inputenc}
\usepackage{amsthm}	
\usepackage{caption}
\usepackage{subcaption}
\usepackage{booktabs}
\usepackage{booktabs}
\usepackage{colortbl}
\usepackage{nameref}
\usepackage{multirow}
\usepackage{animate}
\usetikzlibrary{arrows,decorations.pathmorphing,backgrounds,positioning,fit,matrix, calc, quotes, angles, decorations.pathreplacing}
\usepackage{tikz-cd}
\usepackage{wrapfig}
\tikzset{node distance=2cm, auto}
%PARA el gif
\usepackage{animate}
%Para los comentarios
\usepackage{comment}


  \usetheme{Madrid}       % or try default, Darmstadt, Warsaw, ...
  \usecolortheme{default} % or try albatross, beaver, crane, ...
  \usefonttheme{serif}    % or try default, structurebold, ...
  \setbeamertemplate{navigation symbols}{}
  \setbeamertemplate{caption}[numbered]




\newcommand{\vspan}[1]{\text{span}\left\lbrace #1\right\rbrace}
\newcommand{\dual}[2]{\left\langle #1 , #2\right\rangle}
\def\quotient#1#2{%
    \raise1ex\hbox{$#1$}\big/\hbox{$#2$}%
}

\newcommand{\norm}[1]{\left|\left| #1\right|\right |}

\newcommand{\ultra}[1]{\left( #1\right)_\mathcal{U}}

\newenvironment{variableblock}[3]{%
  \setbeamercolor{block body}{#2}
  \setbeamercolor{block title}{#3}
  \begin{block}{#1}}{\end{block}}
  
  
\graphicspath{{./Figuras/}} 

\usepackage{listings}
\usepackage{color}
\definecolor{dkgreen}{rgb}{0,0.6,0}
\definecolor{gray}{rgb}{0.5,0.5,0.5}
\definecolor{mauve}{rgb}{0.58,0,0.82}

\lstset{ %
  language=R,                     % the language of the code
  basicstyle=\footnotesize,       % the size of the fonts that are used for the code
  numbers=left,                   % where to put the line-numbers
  numberstyle=\tiny\color{gray},  % the style that is used for the line-numbers
  stepnumber=1,                   % the step between two line-numbers. If it's 1, each line
                                  % will be numbered
  numbersep=5pt,                  % how far the line-numbers are from the code
  backgroundcolor=\color{white},  % choose the background color. You must add \usepackage{color}
  showspaces=false,               % show spaces adding particular underscores
  showstringspaces=false,         % underline spaces within strings
  showtabs=false,                 % show tabs within strings adding particular underscores
  frame=single,                   % adds a frame around the code
  rulecolor=\color{black},        % if not set, the frame-color may be changed on line-breaks within not-black text (e.g. commens (green here))
  tabsize=2,                      % sets default tabsize to 2 spaces
  captionpos=b,                   % sets the caption-position to bottom
  breaklines=true,                % sets automatic line breaking
  breakatwhitespace=false,        % sets if automatic breaks should only happen at whitespace
  title=\lstname,                 % show the filename of files included with \lstinputlisting;
                                  % also try caption instead of title
  keywordstyle=\color{blue},      % keyword style
  commentstyle=\color{dkgreen},   % comment style
  stringstyle=\color{mauve},      % string literal style
  escapeinside={\%*}{*)},         % if you want to add a comment within your code
  morekeywords={*,...}            % if you want to add more keywords to the set
} 
